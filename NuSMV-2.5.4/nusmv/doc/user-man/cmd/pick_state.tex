% -*-latex-*-
\begin{nusmvCommand}{pick\_state} {Picks a state from the set of initial states}
 
\cmdLine{pick\_state [-h] [-v] [-r | -i [-a]] [-c "constraints" | -s trace.state]}

Chooses an element from the set of initial states, and makes it the
\code{current state} (replacing the old one). The chosen state is
stored as the first state of a new trace ready to be lengthened by
\code{steps} states by the \command{simulate} command. The state can
be chosen according to different policies which can be specified via
command line options. By default the state is chosen in a
deterministic way.\\
\begin{cmdOpt}
\opt{-v}{Verbosely prints out chosen state (all state and frozen
  variables, otherwise it prints out only the label \code{t.1} of the
  state chosen, where \code{t} is the number of the new trace, that is
  the number of traces so far generated plus one).}

\opt{-r}{Randomly picks a state from the set of initial states.}

\opt{-i}{ Enables the user to interactively pick up an initial state.
  The user is requested to choose a state from a list of possible
  items (every item in the list doesn't show frozen and state
  variables unchanged with respect to a previous item). If the number
  of possible states is too high, then the user has to specify some
  further constraints as ``simple expression''.}

\opt{-a}{ Displays all state and frozen variables (changed and
  unchanged with respect to a previous item) in an interactive
  picking. This option works only if the \commandopt{i} options has
  been specified.}

\opt{-c \parameter{"constraints"}}{ Uses \code{constraints} to
  restrict the set of initial states in which the state has to be
  picked.  \code{constraints} must be enclosed between double quotes
  \code{" "}.}

\opt{-s \parameter{trace.state}}{Picks state from trace.state label. A
  new simulation trace will be created by copying prefix of the source
  trace up to specified state.}

\end{cmdOpt}

\end{nusmvCommand}
